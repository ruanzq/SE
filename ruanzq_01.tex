\documentclass{article}
\usepackage[UTF8]{ctex}
\title{软件工程 作业 01}
\author{阮赵祺}
\begin{document}	
	\maketitle
	\section{git}
	\subsection{关于git}
	git --	The stupid content tracker。Linus Torvalds 是这样给我们介绍 git 的,git是一个非常流行的开源的分布式版本控制系统。
	\subsection{为什么要使用git}
	在软件工程的语境下,版本就是软件的版本。而控制则是相对于手动的自动化控制。软件开发并不是一蹴而就,我所使用的所有软件都在不断的迭代,并无完美的终极版本,停止更新是软件走向衰落和死亡的讯号。软件不断的更新,自然就有无数的版本,你若是使用操作系统附带的资源管理器和文件命名来控制版本,显然太愚蠢了。我需要的是一种更为高级的程序,当然我下达存档命令时,就能自动的记录新的版本,并让我在不同的版本之间快速切换,这就是git。
	\subsection{分布式和集中式}
	git和其他的版本控制系统的核心区别在于git是一种分布式版本控制系统。传统的版本控制系统使用集中式来管理版本,存在单点失败的安全隐患,并且提交新版本需要终端在线,开发环境较为苛刻。而分布式版本控制系统的安全性和开发效率要高很多,因为每个开发者的PC里都有完整的版本库,离线环境下也能正常开发,提交新的版本,等到网络链接可用时再推送到约定的服务器上交换。这里你可能注意到了,git并不是纯粹的分布式版本控制系统,分布式版本控制系统不是银弹,分布式的缺陷在于发现能力极为有限。哪里有可以clone的项目?那里有合作者,这些都是分布式不能解决的问题,所以基于git的Github,正是使用了集中式和分布式各自的优点,从而一举成为最流行的项目托管中心。
	\subsection{如何学习git}
	\subsubsection{初学者请不要试图记住所有东西}
	入门试图记住git的一切是一种非常愚蠢的行为。这条经验不仅仅限于git,其他大部分知识也是同理。git命令繁多,伴随git而生的概念通常较为抽象,背诵式的学习会极大的降低git的使用体验,消磨你的信心。计算机系统结构课程里有两条普适性极强的经验,那就是经常性事件和局部性原则。当这两条经验应用于git学习时,前者揭示了你牢记你经常使用的git命令,所获得的效率远远超过你花大功夫去背诵整个命令集。后者则说明,你当前使用的命令很大几率上会在下次被使用。如果你真的去用过git并做了一些开放,我相信你会认同我的说法,长期使用的只是git命令的一个子集。当然,如果你或精力旺盛无处发泄或立志成为git专家,那你大可不必在意本节内容。
	\subsubsection{网络资源}
	受益于互联网,你可以从网络上取得大量的免费文档,git官网就一个很不错的选择,包含但不限于一本可以做教科书的《Pro Git》,一本命令速查指南,数个视频。
	\subsubsection{使用git做点开发}
	光说不练假把式,要学好git,那你必须使用git进行开发,自嗨项目都行。这样才能才使用中进一步的发现问题,解决问题。如果只是看看命令,那就毫无意义。
	\subsubsection{交流}
	三人行必有我师!
	\section{Github}
	Github是基于git,目前最为流行的开源项目托管平台,当然,如果你付费,你也可以在Github上托管私有项目。你可以从Github上合法的下载大量的开源程序,按所属的协议对这些程序进行增强改动和分发。git是Github的灵魂所在,但Github不仅仅局限于git所提供的功能,Github外延了git的功能,使其不仅仅是一个版本控制系统,更是一个交流平台,你可以通过pull request,issue,评论等方式交流。综上所诉,Github简直是社会主义者和理想主义者的天堂。
	\section{关于三款软件}
	由于本人从未使用过下述软件,对下述软件的直观理解均来自网络文档和交流。Rational Rose是个过气老网红,作为第一代的UML工具,实在是太老旧了,关于Rational Rose的博文都给出了更好的代替软件。Visio是微软推出一款UML工具,算是第二代UML了。如果你觉得下载Visio并取得授权是一件费时费力的,你也可以使用一些在线绘图网站代替,ProcessOn.com就是一个很不错的选择。Power Designer是由华人所开发的UML工具,就样子来说,和Visio差不多,不过Power Designer似乎更擅长对数据库建模,在数据库领域很受推崇。
	\subsection{Microsoft Visio}
	Office Visio 是office软件系列中的负责绘制流程图和示意图的软件,是一款便于IT和商务人员就复杂信息、系统和流程进行可视化处理、分析和交流的软件。使用具有专业外观的 Office Visio 图表,可以促进对系统和流程的了解,深入了解复杂信息并利用这些知识做出更好的业务决策。
	Microsoft Office Visio帮助您创建具有专业外观的图表,以便理解、记录和分析信息、数据、系统和过程。
	大多数图形软件程赖于结构技能。然而,在您使用 Visio 时,以可视方式传递重要信息就像打开模板、将形状拖放到绘图中以及对即将完成的工作应用主题一样轻松。Office Visio 2010中的新增功能和增强功能使得创建 Visio 图表更为简单、快捷,令人印象更加深刻。
	\subsection{PowerDesigner}
	PowerDesigner最初由Xiao-Yun Wang(王晓昀)在SDP Technologies公司开发完成。PowerDesigner是Sybase的企业建模和设计解决方案,采用模型驱动方法,将业务与IT结合起来,可帮助部署有效的企业体系架构,并为研发生命周期管理提供强大的分析与设计技术。PowerDesigner独具匠心地将多种标准数据建模技术(UML、业务流程建模以及市场领先的数据建模)集成一体,并与 .NET、WorkSpace、PowerBuilder、Java™、Eclipse 等主流开发平台集成起来,从而为传统的软件开发周期管理提供业务分析和规范的数据库设计解决方案。此外,它支持60多种关系数据库管理系统(RDBMS)/版本。PowerDesigner运行在Microsoft Windows平台上,并提供了Eclipse插件。
	\subsection{Rational Rose}
	Rational Rose是Rational公司出品的一种面向对象的统一建模语言的可视化建模工具。用于可视化建模和公司级水平软件应用的组件构造。
	\begin{thebibliography}{99}
		\bibitem{}git官方地址 https://git-scm.com/
		\bibitem{}王晓昀\ 百度百科 https://baike.baidu.com/item/%E7%8E%8B%E6%99%93%E6%98%80/9236799
		\bibitem{}git教程 廖雪峰
		\\ https://www.liaoxuefeng.com/wiki/0013739516305929606dd18361248578c67b8067c8c017b000
		\bibitem{}Rational Rose 简介\ 百度百科\ https://baike.baidu.com/item/Rational%20Rose/11019648
		\bibitem{}powerdesigner简介\ 百度百科\ https://baike.baidu.com/item/power%20designer/2482290
		\bibitem{}Microsoft Office Visio简介\ 百度百科 \\ https://baike.baidu.com/item/Microsoft%20Office%20Visio/7180347
		
	\end{thebibliography}
\end{document}