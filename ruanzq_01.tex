\documentclass{article}
\usepackage[UTF8]{ctex}
\title{软件工程 作业 01}
\author{阮赵祺}
\begin{document}
	\maketitle
	\section{git}
	\subsection{关于git}
	git --	The stupid content tracker。Linus Torvalds 是这样给我们介绍 Git 的,Git是一个非常流行的开源的分布式版本控制系统。
	\subsection{关于版本控制}
	在软件工程的语境下,版本就是软件的版本。而控制则是相对于手动的自动化控制。软件开发并不是一蹴而就,通常软件由多个不同分工的人员进行开发,有的时候需要在不同时间点开发的版本中切换,
	\subsection{分布式和集中式}
	git和其他的版本控制系统的核心区别在于git是一种分布式版本控制系统。传统的版本控制系统使用集中式来管理版本,存在单点失败的安全隐患,并且提交新版本需要终端在线,开发环境较为苛刻。而分布式版本控制系统的安全性和开发效率要高很多,因为每个开发者的PC里都有完整的版本库,离线环境下也能正常开发,提交新的版本,等到网络链接可用时再推送到约定的服务器上交换。这里你可能注意到了,git并不是纯粹的分布式版本控制系统,因为没有银弹,分布式版本控制系统的问题在于寻址。哪里有可以clone的项目?那里有合作者,这些都是分布式不能解决的问题,所以基于git的github,正是使用了集中式和分布式各自的优点,从而一举成为最流行的项目托管中心。
	\subsection{学习git的一些心得}
	\subsubsection{不要试图记住所有东西}
	学习git是为了使用git进行版本控制,而不是为了考试。试图记住git的一切是一种非常愚蠢的行为。git命令繁多,自身程序也很复杂,背诵式的学习会极大的降低git的使用体验。就我个人经验来看,大部分时候使用的只是git命令的一个子集,按照CS里面局部性原则,花时间去学习这部分经常使用的子集,就已经足够了。当然,如果你极其聪明,精力旺盛无处发泄或是立志成为git专家,可以忽略本节内容。
	\section{github}
	github是基于git,目前最为流行的开源项目托管平台,当然,如果你付费,你也可以在github上托管私有项目。你可以从github上合法的下载大量的开源程序,这很社会主义。github是基于git的平台,但github不仅仅局限于git,github外延了git的功能,使其不仅仅是一个版本控制系统,更是一个交流平台,你可以通过pull request,issue,评论等方式交流。综上所诉,GitHub简直是社会主义者和理想主义者的天堂。
	\section{Microsoft Visio}
	\section{PowerDesigner}
	\section{Rational Rose}
\end{document}